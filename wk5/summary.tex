% Compiled with Overleaf

\input{preamble}
\input{format}
\input{commands}

\begin{document}

\begin{Large}
    \textbf{Summary: Deslauriers, Louis, et al.}\\
\end{Large}
\break{}
\begin{large}
    \textit{"Measuring actual learning versus feeling of learning in response to being actively engaged in the classroom."}
\end{large}

\pagenumbering{empty}

\vspace{1ex}

\textsf{\textbf{Y4:}} \text{Adam, Alejandro, Ammar, Brendan, Dakshit, Joshua, Marek}\\
\textsf{\textbf{Tutor:}} \text{Dr Mustafa}


\vspace{1ex}
\hrule
\vspace{2ex}

\par This paper concluded that there is a negative correlation between the feeling of learning and actual learning experienced by students in an introductory physics class at university. They measured the feeling of learning by using a short survey at the end of each lecture and actual learning with a multiple-choice test. The researchers calculated that $P < 0.001$ meaning there is a $0.1\%$ chance that this negative correlation is due to random chance. To mitigate the impact of clustering on the experimental results, they randomised two classes and swapped the two teaching styles (active and passive) after the first semester - it was a crossover study. The control variables included the teaching styles of lecturers; the teaching materials provided to students and the average physics background and proficiency between the two classes. This study is significant because it generalises to computer science and can better inform our study method choices. This research paper confirms that increasing cognitive effort through passive learning improves outcomes. This paper is useful because it helps students understand that even if they don't feel like they are learning much (because it feels hard and effortful), they might actually be learning better.
% =================================================

% \newpage

% \vfill

\end{document}

% https://www.overleaf.com/latex/templates/basic-homework-template/vrpzjpnpscbx